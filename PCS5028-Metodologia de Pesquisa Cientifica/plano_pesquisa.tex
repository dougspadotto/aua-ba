\documentclass[
	% -- opções da classe memoir --
	12pt,				% tamanho da fonte
	openright,			% capítulos começam em pág ímpar (insere página vazia caso preciso)
	oneside,			% para impressão em verso e anverso. Oposto a oneside
	a4paper,			% tamanho do papel. 
	% -- opções da classe abntex2 --
	%chapter=TITLE,		% títulos de capítulos convertidos em letras maiúsculas
	%section=TITLE,		% títulos de seções convertidos em letras maiúsculas
	%subsection=TITLE,	% títulos de subseções convertidos em letras maiúsculas
	%subsubsection=TITLE,% títulos de subsubseções convertidos em letras maiúsculas
	% -- opções do pacote babel --
	english,			% idioma adicional para hifenização
	french,				% idioma adicional para hifenização
	spanish,			% idioma adicional para hifenização
	brazil,				% o último idioma é o principal do documento
	]{abntex2}

% ---
% PACOTES
% ---

% ---
% Pacotes fundamentais 
% ---
\usepackage{lmodern}			% Usa a fonte Latin Modern
\usepackage[T1]{fontenc}		% Selecao de codigos de fonte.
\usepackage[utf8]{inputenc}		% Codificacao do documento (conversão automática dos acentos)
\usepackage{indentfirst}		% Indenta o primeiro parágrafo de cada seção.
\usepackage{color}				% Controle das cores
\usepackage{graphicx}			% Inclusão de gráficos
\usepackage{microtype} 			% para melhorias de justificação
% ---
% ---
% Pacotes de citações
% ---
\usepackage[brazilian,hyperpageref]{backref}	 
% Paginas com as citações na bibl
\usepackage[alf]{abntex2cite}	% Citações padrão ABNT

\titulo{Organização e instrumentação de uma cadeia logística  de verduras e legumes orgânicos na cidade de São Paulo}
\autor{Douglas Spadotto}
\local{São Paulo}
\data{2018}
\instituicao{%
  Universidade de São Paulo
  \par
  Escola Politécnica da Universidade de São Paulo
  \par
  Programa de Pós-Gradução em Engenharia de Sistemas Logísticos}
\tipotrabalho{Projeto de pesquisa}
\preambulo{Projeto de pesquisa elaborado durante a disciplina de Metodologia de Pesquisa Científica em Engenharia de Computação.}
% ---

% ---
% Configurações de aparência do PDF final

% alterando o aspecto da cor azul
\definecolor{blue}{RGB}{41,5,195}

% informações do PDF
\makeatletter
\hypersetup{
     	%pagebackref=true,
		pdftitle={\@title}, 
		pdfauthor={\@author},
    	pdfsubject={\imprimirpreambulo},
	    pdfcreator={LaTeX with abnTeX2},
		pdfkeywords={abnt}{latex}{abntex}{abntex2}{projeto de pesquisa}, 
		colorlinks=true,       		% false: boxed links; true: colored links
    	linkcolor=blue,          	% color of internal links
    	citecolor=blue,        		% color of links to bibliography
    	filecolor=magenta,      		% color of file links
		urlcolor=blue,
		bookmarksdepth=4
}
\makeatother
% --- 
% O tamanho do parágrafo é dado por:
\setlength{\parindent}{1.3cm}

% Controle do espaçamento entre um parágrafo e outro:
\setlength{\parskip}{0.2cm}  % tente também \onelineskip

\begin{document}

% Retira espaço extra obsoleto entre as frases.
\frenchspacing 

\imprimircapa
\imprimirfolhaderosto

\chapter*{Resumo}
A demanda por produtos orgânicos cresce mundialmente porém no Brasil o crescimento é inferior ao restante do mundo devido a custos elevados, muitos deles de natureza logística. A estruturação de uma cadeia logística de produtos agrícolas utilizando um modelo pode trazer eficiências, bem como o uso de tecnologias de sensores e os dados que produzem. 
Este trabalho visa escolher e aplicar um modelo da literatura a uma cadeia logística de orgânicos existente e identificar lacunas e oportunidades de melhoria. Também objetiva-se o uso de sensores de IoT para a medição da cadeia.

\textbf{Palavras-chave}: cadeia de suprimentos,produtos agrícolas orgânicos, IoT

\chapter*{Abstract}
The demand for organic products is growing globally, however in Brazil the growth is slower to the rest of the world due to high costs, many of them of due to logistics. Structuring a supply chain of agricultural products using a model can bring efficiencies, as well as the use of sensor technologies and the data they produce.
This paper aims to choose and apply an appropriate model from the literature to an existing organic supply chain and identify gaps and opportunities for improvement. The objective is also to use IoT sensors to measure the chain wherever applicable.

\textbf{Keywords}: supply chain, organic produce, IoT

\chapter{Introdução e justificativa}

Nas últimas décadas, segundo \citeonline{Campanhola2001}, demonstrou-se que produtos orgânicos possuem características próprias de cultivo, colheita, armazenamento e  transporte. Além disso, o cultivo destes produtos traz benefícios para o produtor. Entre eles, menores gastos com insumos e melhor retorno financeiro na comercialização. 

O interesse na compra desta produção é crescente da parte dos consumidores, conforme \citeonline{Tsolakis2014}. O autor indica como principal motivação ao consumo de produtos orgânicos a percepção de que estes são mais saudáveis, saborosos e seguros.

Contudo, o crescimento do consumo de produtos orgânicos no Brasil é inferior à média mundial, devido a problemas de logísticos, como custos elevados de transporte, de conservação e difícil acesso aos produtos, de acordo com o relatório publicado pelo Conselho Brasileiro da Produção Orgânica e Sustentável (\citeonline{ORGANIS2017}). 

A imposição de uma estrutura na organização de uma cadeia logística de alimentos de origem agrícola tem o objetivo de torná-la competitiva e sustentável. Uma forma de estruturação é discutida em \citeonline{Tsolakis2014}, que apresenta uma estrutura hierárquica tanto dos atores da cadeia de suprimentos quanto um arcabouço para tomada de decisão do seu gerenciamento.

A dissertação de \citeonline{Silva2010} também discute uma estrutura de sistema logístico, neste caso para a cadeia fria, que possui características que se interssecciona com a cadeia de orgânicos, devido a perecibilidade dos produtos. 

Outro trabalho relacionado a este tema é o de \citeonline{COSTA2006}, que analisa estratégias de gestão para a manga produzida no Brasil para exportação. O autor também relaciona estas estratégias a estruturas de cadeias logísticas da literatura.

A instrumentação de uma cadeia logística pode ser utilizada para atingir níveis de serviço mais elevados, porém a medida de quanto uma cadeia logística, ou até mesmo diferentes produtos dentro de uma mesma cadeia, deve ser instrumentada difere de acordo com o retorno sobre o investimento e viabilidade, como explica \citeonline{Dabbene2014}.

Tendo em vista este cenário e os trabalhos já publicados, o desenvolvimento da pesquisa apresentada neste projeto se justifica sobretudo pela necessidade de estudos que contribuam para a mitigação dos custos elevados e das ineficiências da cadeia logística de produtos orgânicos.

\chapter{Objetivos}

Esta pesquisa tem como objetivo geral mapear uma cadeia logística de frutas, verduras e legumes orgânicos existente na cidade de São Paulo.

Entre os objetivos específicos, pretendemos identificar os atores desta cadeia, seus papéis e responsabilidades e verificar as formas como decisões são tomadas por parte dos produtores, organizadores e consumidores.

Em um segundo momento, relacionaremos em cada elo desta cadeia   opções de intrumentação com sensores para coleta de dados de localização, tempo e características que afetam a qualidade dos produtos (temperatura, umidade). Identificaremos então como estes dados se relacionam com sistemas de rastreamento em uso - códigos de barras e QR (Quick Response), de acordo com padrões ISO/International Electro-technical Commission (IEC) 15961, 15962, 24791, 15459, 15418 e 15434.

Por fim, também mostraremos como a aplicação destes sensores pode ser utilizada para o atendimento do padrão ISO para cadeias de suprimentos de alimentos, o ISO 22005:2007 \citeonline{international2007traceability}.


\chapter{Plano de trabalho}

Este projeto tem duração prevista de 20 meses, divididos em 5 quadrimestres, distribuídos da seguinte forma:

\begin{itemize}
\item Quadrimestre 1: de Setembro/2018 a Dezembro/2018.
\item Quadrimestre 2: de Janeiro/2019 a Abril/2019.
\item Quadrimestre 3: de Maio/2019 a Agosto/2019.
\item Quadrimestre 4: de Setembro/2019 a Dezembro/2019.
\item Quadrimestre 5: de Janeiro/2020 a Abril/2020.
\end{itemize}
Dentro deste prazo, realizaremos as seguintes atividades:

\begin{enumerate}
\item Comparativo de modelos de cadeia logística para produtos agrícolas \textit{in natura};

\item Listagem de componentes de IoT aplicáveis a esta cadeia logística: classes e modelos de equipamentos,  protocolos de comunicação e análise de viabilidade para aplicação no estudo;

\item Qualificação do plano de pesquisa;

\item Elaboração de questionários e planos de visitas a campo;

\item Contato com cadeia logística para coleta de dados: 
\begin{enumerate}
	\item Realização de reuniões de coleta de dados com organizadores;
	\item Visitas a campo: coleta de dados sobre plantio, colheita, estoque e entregas a coletivos;
	\item Visitas a coletivos de consumo: coleta de dados sobre recebimento, armazenamento temporário e entrega final;
	\item Entrevistas ou questionários com consumidores finais.
\end{enumerate}

\item Análise dos dados coletados;

\item Escrita de dissertação;

\item Defesa de dissertação.
\end{enumerate}

\section{Cronograma de execução}

A tabela \ref{tabela:1} estrutura as atividades como linhas e os quadrimestres como colunas.

\begin{table}[h!]
\begin{center}
 \begin{tabular}{|| c || c | c | c | c | c | c ||} 
 \hline
 Atividade  & 1 & 2 & 3 & 4 & 5 \\ [0.5ex] 
 \hline\hline
 1 & X & & & &  \\ 
 \hline
 2 & X & & & &  \\
 \hline
 3 & & X &  & & \\
 \hline
 4 & & X & X & & \\
 \hline
 5 & & X & X & & \\ 
 \hline
 6 & & & X & & \\ 
 \hline
 7 & & & & X & X \\
 \hline
 8 & & & & & X \\ [1ex] 
 \hline
\end{tabular}
\end{center}
\caption{Cronograma de execução}
\label{tabela:1}
\end{table}

\chapter{Material e métodos}

Para realizar o mapeamento da cadeia proposto, coletaremos dados de produtores rurais, organizadores, pontos de coleta e consumidores. A cadeia logística a ser analisada será a Banca Orgânica, do Instituto Auá. Esta cadeia une coletivos de consumo auto-organizados com produtores de produtos agrícolas orgânicos e agroecológicos.

Além de características físicas como capacidade produtiva e  localização dos produtores, centros de coleta e consumidores, aplicaremos questionários aos envolvidos em cada um destes pontos da cadeia.

Documentaremos a produção, organização e distribuição através de visitas a cada ponto da cadeia de distribuição. Estas visitas seguirão um roteiro a ser elaborado previamente. No caso de falta de dados em uma visita devido a falhas no roteiro, este será reformulado e faremos novas visitas utilizando o roteiro corrigido.

Na impossibilidade de acesso a esta rede, existem outras cadeias que podem ser estudadas. Por exemplo, a dissertação de \citeonline{Terrazzan2009} utilizou como fonte redes de supermercados na cidade de São Paulo. Sem acesso poderemos coletar dados públicos sobre o mercado de orgânicos.

Coletaremos dados analíticos da cadeia logística através de sensores que serão selecionados no decorrer do projeto. Os dados coletados por estes sensores, que serão instalados nos veículos e caixas de entrega, fará parte de um banco de dados que será utilizado para as análises.

\chapter{Forma de análise dos resultados}

Agruparemos os resultados da coleta de dados em dimensões de demografia, localização, características de produção, distribuição e consumo.

Faremos comparações de dados de produtores e consumidores para identificação de eventuais assimetrias entre oferta e demanda de determinados produtos. Também procuraremos estudar a relação entre localização de produtores e centros de coleta, e entre centros de coleta e consumidores.

Indicaremos oportunidades de otimização da cadeia logística, visando reduzir custos e tornar os produtos e meio de entrega mais atrativos para atender a futuras demandas: minimizar o tempo entre colheita e consumo, eliminar intervenções e processos manuais para integração da cadeia logística entre produtor e consumidor.

Do ponto de vista de melhoria do nível de serviço, buscaremos indicar formas de ajustar roteiros e entregas que atendam as necessidades dos consumidores.

% ----------------------------------------------------------
% Referências bibliográficas
% ----------------------------------------------------------
\bibliography{projeto}

\end{document}
